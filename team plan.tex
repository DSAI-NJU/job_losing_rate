\documentclass[aspectratio=169,11pt]{beamer}
\usetheme{Madrid}
\usecolortheme{default}
\setbeamertemplate{navigation symbols}{}
\setbeamertemplate{itemize items}[circle]

\usepackage[T1]{fontenc}
\usepackage[utf8]{inputenc}
\usepackage{lmodern}
\usepackage{amsmath, amssymb}
\usepackage{graphicx}
\usepackage{booktabs}
\usepackage{hyperref}

\makeatletter
\setbeamertemplate{footline}{%
	\leavevmode%
	\hbox{%
		\begin{beamercolorbox}[wd=.62\paperwidth,ht=2.25ex,dp=1ex,leftskip=2ex]{title in head/foot}%
			\usebeamerfont{title in head/foot}\insertshorttitle
		\end{beamercolorbox}%
		\begin{beamercolorbox}[wd=.38\paperwidth,ht=2.25ex,dp=1ex,rightskip=2ex]{date in head/foot}%
			\usebeamerfont{date in head/foot}\insertshortdate{}\hspace{1em}%
			\insertframenumber{} / \inserttotalframenumber
		\end{beamercolorbox}%
	}%
	\vskip0pt%
}
\makeatother

% ---- Meta ----
\title[Employment \& Unemployment (Baidu Index \& Baike)]%
{Midterm Research Plan: Employment \& Unemployment (Baidu Index \& Baidu Encyclopedia)}
\author{Chenxi Zhang, Haowen Shi, Haotian Zhou}
\institute{Course: Data Science and AI}
\date[Nov 27, 2025]{Nov 27, 2025}

\begin{document}
	
	% 1) Title
	\begin{frame}
		\titlepage
	\end{frame}
	
	% 2) Executive Summary
	\begin{frame}{Executive Summary}
		\small
		\textbf{Aim:} Use Baidu Index (search interest) and Baidu Encyclopedia to identify employment market trends, influencing factors, and potential patterns. \\
		\textbf{Scope:} China, past five years (2019--2025). \\
		\textbf{Deliverables:} Clean datasets, descriptive analytics, and clear visuals.
		\medskip
		
		\textbf{Pipeline}
		\begin{itemize}
			\item Data Collection (Baidu Index \& Baidu Encyclopedia)
			\item Data Cleaning (completeness, missingness, duplicates, outliers)
			\item Descriptive Analysis (distribution, correlation, trends)
			\item Visualization (time trends, regional comparisons, correlations)
			\item Conclusions \& Outlook
		\end{itemize}
	\end{frame}
	
	% 3) Motivation / Significance
	\begin{frame}{0. Motivation and Significance}
		\small
		\textbf{Why search data?}
		\begin{itemize}
			\item Search interest provides a high-frequency proxy for public concern and job-market sentiment.
			\item Compared to official statistics (monthly/quarterly), it reacts faster to shocks and policy events.
		\end{itemize}
		\medskip
		
		\textbf{Practical value}
		\begin{itemize}
			\item Early-warning signals of labor-market stress (graduation season, layoffs, policy shifts).
			\item Support regional employment monitoring and policy evaluation.
		\end{itemize}
	\end{frame}
	
	% 4) Research Background and Objectives
	\begin{frame}{1. Research Background and Objectives}
		\small
		\textbf{Background:} Employment and unemployment are core issues affecting social stability and economic development. \\
		\textbf{Objective:} Use Baidu Index and Baidu Encyclopedia to identify employment-market trends, influencing factors, and potential patterns.
		\medskip
		
		\textbf{Key questions}
		\begin{itemize}
			\item What do search-interest dynamics reveal about temporal and regional variations?
			\item How do related concepts (e.g., unemployment rate, job hunting) co-move?
			\item Which policy topics or definitions appear most frequently in encyclopedia entries?
		\end{itemize}
	\end{frame}
	
	% 5) Related Work / Contribution
	\begin{frame}{1.1 Related Work and Our Contribution}
		\small
		\textbf{Prior evidence}
		\begin{itemize}
			\item Search data has been used to nowcast macro variables (consumption, finance, epidemics).
			\item Labor-market studies show online attention correlates with job-search intensity.
		\end{itemize}
		\medskip
		
		\textbf{Our contribution}
		\begin{itemize}
			\item Combine \textit{Index (behavior)} + \textit{Baike (definition/policy text)}.
			\item Provide interpretable descriptive results by region/keyword/event.
			\item Build a reproducible dataset pipeline for future forecasting extensions.
		\end{itemize}
	\end{frame}
	
	% 6) Data Collection — Baidu Index
	\begin{frame}{2. Data Collection — 2.1 Baidu Index}
		\small
		\textbf{Keywords:} Select terms related to “employment” and “unemployment” (e.g., “job hunting”, “unemployment rate”). \\
		\textbf{Method:} Use Baidu Index self-service tools to set time window and regional scope, then export data.
		\medskip
		
		\textbf{Expected fields}
		\begin{itemize}
			\item \texttt{date}, \texttt{region}, \texttt{keyword}
			\item \texttt{search\_index\_total}, \texttt{pc\_index}, \texttt{mobile\_index}
			\item frequency (daily/weekly), notes
		\end{itemize}
		\textbf{Coverage:} National and major provinces.
	\end{frame}
	
	% 7) Keyword Strategy
	\begin{frame}{2.1.1 Keyword Selection Strategy}
		\small
		\textbf{Principles}
		\begin{itemize}
			\item \textbf{Coverage:} include both outcome terms (``unemployment'') and behavior terms (``job hunting'').
			\item \textbf{Specificity:} avoid overly broad words; prefer policy/market-relevant phrases.
			\item \textbf{Robustness:} add synonyms and colloquial variants to reduce wording bias.
		\end{itemize}
		\medskip
		
		\textbf{Candidate buckets}
		\begin{itemize}
			\item Employment: ``employment'', ``recruitment'', ``campus hiring'', ``graduate jobs''
			\item Unemployment: ``unemployment'', ``unemployment rate'', ``layoff'', ``jobless''
			\item Job-search behavior: ``job hunting'', ``resume'', ``interview'', ``offer''
		\end{itemize}
	\end{frame}
	
	% 8) Data Collection — Baidu Encyclopedia
	\begin{frame}{2. Data Collection — 2.2 Baidu Encyclopedia}
		\small
		\textbf{Targets:} Entries like “employment policies” and “unemployment types.” \\
		\textbf{Extraction:} Policy details, industry employment data, unemployment causes.
		\medskip
		
		\textbf{Typical fields}
		\begin{itemize}
			\item entry title, abstract, section text
			\item key dates (publication/revision), policy highlights
			\item target groups, administrative level, references, URL
		\end{itemize}
		\textbf{Method:} Web scraping with standard HTML parsing; store as structured JSON.
	\end{frame}
	
	% 9) Text Processing Plan
	\begin{frame}{2.2.1 Baike Text Processing Plan}
		\small
		\textbf{Goals}
		\begin{itemize}
			\item Extract policy/definition themes and their evolution over time.
		\end{itemize}
		\medskip
		
		\textbf{Planned steps}
		\begin{itemize}
			\item Cleaning: remove boilerplate, unify punctuation, Chinese tokenization.
			\item Keyword/phrase mining: TF-IDF or simple frequency counts.
			\item Topic sketching (descriptive): cluster sections by similarity to find theme groups.
		\end{itemize}
		\medskip
		
		\textbf{Output}
		\begin{itemize}
			\item Policy-topic frequency table by year.
			\item Co-occurrence network of policy terms (descriptive).
		\end{itemize}
	\end{frame}
	
	% 10) Data Cleaning (part 1)
	\begin{frame}{3. Data Cleaning (Part 1)}
		\small
		\textbf{Initial Review:} Check data completeness and accuracy. \\
		\textbf{Missing Values:} Fill with mean/linear interpolation, or delete invalid entries; flag imputed points. \\
		\textbf{Duplicates:} Remove repeated records and keep an audit trail.
		\medskip
		
		\textbf{Outputs}
		\begin{itemize}
			\item Cleaned index table(s) aligned by date, region, and keyword
			\item Structured encyclopedia table(s) for downstream analysis
		\end{itemize}
	\end{frame}
	
	% 11) Data Cleaning (part 2)
	\begin{frame}{3. Data Cleaning (Part 2)}
		\small
		\textbf{Outliers:} Identify via \(3\sigma\), IQR fences, or boxplot indicators. \\
		\textbf{Format Conversion:} Standardize date/numeric formats; ISO-8601 dates, normalized region codes.
		\medskip
		
		\textbf{Quality flags}
		\begin{itemize}
			\item \texttt{impute\_flag}, \texttt{outlier\_flag}, \texttt{source\_note}
			\item reproducible scripts/notebooks with deterministic results
		\end{itemize}
	\end{frame}
	
	% 12) Descriptive Analysis — Distribution
	\begin{frame}{4. Descriptive Analysis — Distribution}
		\small
		\textbf{Goal:} Analyze distributions across regions and time to find central tendencies and dispersion.
		\medskip
		
		\textbf{Examples}
		\begin{itemize}
			\item Regional boxplots/violin plots for key keywords
			\item Temporal distribution summaries (by month/quarter/year)
			\item Heatmaps of average index levels or coefficients of variation
		\end{itemize}
	\end{frame}
	
	% 13) Descriptive Analysis — Correlation
	\begin{frame}{4. Descriptive Analysis — Correlation}
		\small
		\textbf{Goal:} Explore links between employment/unemployment search data and economic factors, and among related keywords.
		\medskip
		
		\textbf{Examples}
		\begin{itemize}
			\item Correlation matrices (Pearson/Spearman)
			\item Scatter plots with trend lines (e.g., ``layoff'' vs ``job hunting'')
			\item Optional alignment with macro indicators (GDP, CPI, youth unemployment)
		\end{itemize}
	\end{frame}
	
	% 14) Descriptive Analysis — Trends
	\begin{frame}{4. Descriptive Analysis — Trends}
		\small
		\textbf{Goal:} Use time-series views to observe changes and discuss short-term signals.
		\medskip
		
		\textbf{Examples}
		\begin{itemize}
			\item Line charts of keyword indices with moving averages
			\item Seasonal/holiday/graduation-season annotations
			\item Descriptive STL decomposition or change-point checks
		\end{itemize}
	\end{frame}
	
	% 15) Event Annotation / External Context
	\begin{frame}{4.1 Event Annotation and External Context}
		\small
		\textbf{Why annotate events?}
		\begin{itemize}
			\item Peaks are often driven by real-world shocks or policies.
		\end{itemize}
		\medskip
		
		\textbf{Planned annotations}
		\begin{itemize}
			\item Graduation seasons (Jun--Jul annually)
			\item Major policy releases (employment support, labor market reforms)
			\item Macro shocks (COVID waves, large-scale layoffs, tech/real estate cycles)
		\end{itemize}
		\medskip
		
		\textbf{Use}
		\begin{itemize}
			\item Compare pre/post-event search levels; describe regional heterogeneity.
		\end{itemize}
	\end{frame}
	
	% 16) Visualization
	\begin{frame}{5. Data Visualization}
		\small
		\textbf{Tools:} Python (Matplotlib, Seaborn) or BI tools (e.g., FineBI). \\
		\textbf{Charts:} Line charts, bar charts, scatter plots, heatmaps, co-occurrence networks.
		\medskip
		
		\textbf{Design guidelines}
		\begin{itemize}
			\item Consistent labels; figures include source \& study window
			\item Clear legends/annotations; readable fonts for classroom screens
			\item Reproducible and parameterized code
		\end{itemize}
	\end{frame}
	
	% 17) Robustness / Limitations
	\begin{frame}{5.1 Robustness and Limitations}
		\small
		\textbf{Robustness checks (descriptive)}
		\begin{itemize}
			\item Compare multiple synonyms to confirm consistent patterns.
			\item PC vs Mobile split to see user-group differences.
			\item Smoothing sensitivity (MA window length).
		\end{itemize}
		\medskip
		
		\textbf{Limitations}
		\begin{itemize}
			\item Search data is attention, not necessarily real employment outcomes.
			\item Media/viral events may cause temporary spikes.
			\item Baike entries may have editorial lag or incomplete regional detail.
		\end{itemize}
	\end{frame}
	
	% 18) Timeline
	\begin{frame}{6. Project Timeline (Next 4--5 Weeks)}
		\small
		\begin{itemize}
			\item \textbf{Week 1:} finalize keyword list; collect Index \& Baike data; build raw database.
			\item \textbf{Week 2:} cleaning pipeline; missing/outlier handling; unify region codes.
			\item \textbf{Week 3:} descriptive analysis (distribution/correlation/trends); event annotations.
			\item \textbf{Week 4:} visualization polishing; interpret results; draft report/slides.
			\item \textbf{Week 5 (if time):} optional extension: simple forecasting / policy comparison.
		\end{itemize}
	\end{frame}
	
	% 19) Team Roles
	\begin{frame}{6.1 Team Roles and Work Allocation}
		\small
		\begin{itemize}
			\item \textbf{Chenxi Zhang:} Baidu Index keyword design; data export; regional comparison analysis.
			\item \textbf{Haowen Shi:} scraping/structuring Baike text; text mining + topic sketches.
			\item \textbf{Haotian Zhou:} cleaning scripts; time-series visualization; robustness checks.
		\end{itemize}
		\medskip
		
		\textbf{Collaboration}
		\begin{itemize}
			\item shared Git repo; weekly merge + clear data/version logs.
		\end{itemize}
	\end{frame}
	
	% 20) Risk & Mitigation
	\begin{frame}{6.2 Risks and Mitigation}
		\small
		\textbf{Potential risks}
		\begin{itemize}
			\item Data access limits / missing regions.
			\item Changes in Baidu Index export rules.
			\item Baike pages with inconsistent templates.
		\end{itemize}
		\medskip
		
		\textbf{Mitigation}
		\begin{itemize}
			\item Backup keyword sets; cache raw HTML locally.
			\item Use modular scrapers; log failures and retry.
			\item Fall back to partial-region analysis with clear caveats.
		\end{itemize}
	\end{frame}
	
	% 21) Conclusions & Outlook
	\begin{frame}{7. Conclusions and Outlook}
		\small
		\textbf{Expected conclusions}
		\begin{itemize}
			\item Summarize distribution/correlation/trend signals by keyword and region.
			\item Highlight data-driven value of search-interest proxies for labor-market research.
		\end{itemize}
		\medskip
		
		\textbf{Outlook}
		\begin{itemize}
			\item Expand sources (social media, job-posting platforms).
			\item Add short-term forecasting or deeper policy-event causal comparisons.
		\end{itemize}
	\end{frame}
	
\end{document}
