\documentclass[12pt]{article}
\usepackage{fontspec}
\usepackage{xeCJK}
\setCJKmainfont{STSong}
\usepackage{geometry}
\geometry{a4paper, margin=2.5cm}
\usepackage{hyperref}
\title{用于监测社会就业/失业状况的关键词列表与分类说明}
\author{(你的名字/机构)}
\date{\today}

\begin{document}

\maketitle

\section*{前言}

在研究社会就业与失业状况时,仅依赖官方统计数据(如就业率、失业率、劳动力参与率等)容易遗漏个体行为与心理层面的波动。通过分析网络搜索关键词(search keywords)——尤其是求职/失业相关的高频词 —— 能更及时、敏感地捕捉民间的就业压力、再就业意愿、经济焦虑、劳动力供需变化等微观信号。因此,本报告整理了一组关键词(及其分类与选择理由),供作为 “就业/失业压力指数” 的基础参考。

\section{关键词列表与分类}

下面将关键词按 “目的/含义/反映维度” 分为若干类别。

\subsection*{就业 — 求职行为与就业市场活跃度}

\begin{itemize}
  \item 找工作  
  \item 求职  
  \item 招聘  
  \item 招聘信息  
  \item 校招 / 春招 / 秋招  
  \item 应届生就业  
  \item 面试  
  \item 简历  
\end{itemize}

\textbf{选择理由}:

\begin{itemize}
  \item 这些词直接反映了个体的主动求职行为与市场对劳动力的需求 —— 搜索“招聘”“求职”“找工作”说明至少有一部分人群在积极寻找工作,市场可能存在流动性/岗位供给。  
  \item “校招/春招/秋招”与“应届生就业”可以捕捉每年高校毕业生集中进入劳动力市场的结构性阶段,有助于识别毕业季/青年就业压力。  
  \item “面试”“简历”等关键词则反映求职过程的深层环节,不仅是“看看有没有工作”,而是“真的准备投递/应聘”。  
\end{itemize}

\subsection*{就业 — 就业压力/就业困难}

\begin{itemize}
  \item 就业难  
  \item 就业形势  
  \item 应届生找工作难  
  \item 行业前景  
  \item 招聘网站(或“招聘平台”)  
\end{itemize}

\textbf{选择理由}:

\begin{itemize}
  \item 当“就业难”“就业形势”“找工作难”这些关键词搜索量上升时,往往意味着求职者在感受到就业市场竞争激烈、岗位匮乏,或者对未来就业前景不乐观 —— 这是就业压力与社会情绪的重要信号。  
  \item “行业前景”/“招聘网站”之类词,也可能反映结构性行业变化/人们转型意愿。  
\end{itemize}

\subsection*{失业/裁员/再就业压力}

\begin{itemize}
  \item 失业  
  \item 裁员  
  \item 裁员潮  
  \item 优化(作为“被裁”“裁员”的委婉表达)  
  \item N+1(裁员补偿相关)  
  \item 失业金 / 失业补助 / 失业保险  
  \item 失业登记  
  \item 再就业  
  \item 低门槛工作 / 临时工 / 蓝领招聘  
\end{itemize}

\textbf{选择理由}:

\begin{itemize}
  \item 这些关键词反映劳动力被动状态 —— 由就业转为失业、裁员、经济补助需求、社会保障需求等,是就业市场供给收缩或个人经济压力的信号。  
  \item “再就业”“低门槛工作”“临时工/蓝领招聘”等词可以反映失业者为了生计而尝试重新进入劳动市场,或转向门槛较低、流动性较大的岗位。  
  \item 同时,这类关键词也可能反映经济周期、产业变动、结构性失业等宏观冲击。  
\end{itemize}

\subsection*{结构性/人群/制度相关 — 特定群体的就业压力}

\begin{itemize}
  \item 35岁就业(或“35岁找工作”)  
  \item 公务员考试 / 国考 / 事业单位考试  
  \item 考研 vs 就业  
  \item 外卖骑手 / 蓝领招聘 / 底层劳动岗位  
\end{itemize}

\textbf{选择理由}:

\begin{itemize}
  \item “35岁就业”反映了中年/中青年劳动力群体在职场年龄限制、转型困难等结构性压力。  
  \item “公务员考试/事业单位考试”在中国为很多人眼中的“铁饭碗”,当这类关键词热度上升时,可能意味着市场上普通岗位/私企岗位缺乏吸引力、竞争激烈。  
  \item “考研 vs 就业”体现高等教育扩招、学历通胀下许多毕业生暂缓就业、转向继续深造的选择。  
  \item “外卖骑手/蓝领/底层劳动岗位”相关词则能反映社会底层/非正规就业对失业人群的吸纳能力,是社会结构性就业压力的重要信号。  
\end{itemize}

\section{关键词使用说明与方法建议}

\begin{itemize}
  \item 建议将上述关键词分组后,分别统计各组在搜索引擎(如百度/谷歌等)的月度/季度搜索指数,以构造多个子指数(例如“求职热度指数”“失业压力指数”“结构性就业焦虑指数”等)。  
  \item 通过对比、趋势分析,可以观察就业市场的周期性波动 —— 例如每年毕业季、年初/年末招聘/裁员高峰、经济冲击期等。  
  \item 注意区分“就业行为类”(主动求职/跳槽)与“失业/被动类”(裁员、再就业、补助申请等),因为它们反映的是劳动力市场不同方向的流动与压力。  
  \item 对于结构性/人群/制度相关关键词,可以进一步辅以人口统计数据、学历分布、行业分布等进行分层分析,以识别不同群体(毕业生、中年、低学历、底层劳动者等)所面临的差异化就业压力。  
\end{itemize}

\section*{结语}

通过上述关键词及其分类,我们可以构建一个比较灵敏、微观、反映真实社会就业/失业状况的“网络搜索指数体系”。这种方法对研究就业市场周期、结构性问题、社会情绪与劳动者行为变化尤其有价值。希望这个列表/分类对你的报告/论文/项目有所帮助。

\end{document} 